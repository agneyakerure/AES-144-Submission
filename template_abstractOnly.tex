% Template for 61st conference for non-peer-reviewed articled
\documentclass[convention]{aesconf}

% Graphics path
\graphicspath{{./}{figures/}}

% UTF-8 encoding is recommended but use one that works for you.
\usepackage[utf8]{inputenc}

% Highly recommended package for better looking text automatically.
\usepackage{microtype}

% Natbib is used for more control on citations. You can use other moderd
% bibliography packages but please try to match the provided style.
\usepackage[numbers,square]{natbib} 


% These are useful for different purposes.
\usepackage{color}
\usepackage{url}


% The full title of the paper
\title{Audio Source Localization as an Input to Virtual Reality Systems}

% Put the authors in order here. The number in brackets define the corresponding affiliation.
\author[1]{Agneya A. Kerure}
\author[1]{Jason Freeman}

% Affiliations go here
\affil[1]{Georgia Institute of Technology}

% Correspondece should include the corresponding author's name and e-mail address
\correspondence{Agneya A. Kerure}{kerure.agneya@gatech.edu}

% These are used for headers. Anything that fits is okay. Please use proper punctuation.

% If there are many authors, please use the form "First author et al."
\lastnames{Kerure and Freeman}

% Short title should describe your topic but not be too long.
\shorttitle{Audio Source Localization in VR}


% This is required and draws the top title
\include{aestitle}

\begin{document}


\twocolumn[
\maketitle % MANDATORY! 

\begin{onecolabstract}
This paper details an effort towards incorporating audio source localization as an input to virtual reality systems, focusing primarily on games. The goal of this research is to find a novel method to use live audio as an input for level generation or creation of elements and objects in a virtual reality environment. The paper discusses the current state of audio-based games and virtual reality, and details the design requirements of a system consisting of a circular microphone array which can be used to localize the input audio. The paper also briefly discusses signal processing techniques used for audio information retrieval, and introduces a prototype of an asymmetric virtual reality first-person shooter game as proof-of-concept of the potential of audio source localization for augmenting the immersive nature of virtual reality.
\end{onecolabstract}
]

\section{Introduction}
Audio and music have historically been a passive part of video games. Soundtracks that accompany video games are used to captivate the player's attention and incorporate a sense of realism to the game environment, but they have rarely been an active part of the game-play mechanism until recently.
Music driven games started out in the form of rhythm based games where the player's sense of rhythm is challenged. These games typically involve a player in performing certain actions in the form of a dance or pressing buttons in a sequence. With the advent of virtual reality, 
Virtual reality rhythm games mostly work in a similar manner but rely on hand held controllers or motion trackers to interact with the rhythmic elements of the game.
Apart from rhythm, there are a few games which involve audio from the user as a core mechanism of game-play. These typically rely on features like volume 
 
 - what you propose - why in VR - why audio
 

\section{Background}

- first rhythm game example

- popular rhythm games example

- audio based games example - chicken scream

- VR rhythm games example

\section{System}  

- brief overview of how we localize sounds and localization systems/techniques

- how circular microphone array does it

- Pitch Detection

- Frequency band detection 

\subsection{Tools} 
Is this section even important??
HTC Vive
Respeaker mic array
juce
unity3d

\section{Game Design} 

\subsection{Mechanics and UI} 

\section{Feedback} 

\section{Challenges} 

\section{Future Work} 

Summarize your work and conclude.

Example cites, \citet{Pulkki1997:VBAPbase} created VBAP and DirAC \citep{Pulkki2007:DirAC_JAES}.

\bibliographystyle{jaes}

% Reference to bibliography file.
\bibliography{refs}


\end{document}